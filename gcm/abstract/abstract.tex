\documentclass[a4paper,10pt]{article}
%\documentclass[a4paper,10pt]{scrartcl}

\usepackage[utf8]{inputenc}
\usepackage{cite}


\title{\Huge {\textbf {Push Notifications using Google Cloud Messaging}}}
\author{
	\Large { Kevin Joseph } \\
	josephkevin1995@gmail.com \\
	\small Roll No 33, S7 CSE \\
	\\
	\Large {Guide : Vipin Vasu A V}\\
	vipin@cet.ac.in
}
\date{\today}

\begin{document}
	\maketitle
	\nocite{*}

	\renewcommand{\abstractname}{\Large Abstract}
	\begin{abstract}
		Android traditionally kept data synchronization between android-device and server-side using a method of pullling. Each 
		Android device has to poll the server for updated data, which leads to unneccessary network traffic
		and wastage of device battery. In order to overcome this weakness, a data pushing service, GCM was introduced.
		Push describes a style of communication where the request for a given transaction is initiated by the publisher
		or central server. Push messaging is a multi-channel mobile cloud communication platform that unifies push
		notifications, SMS and instant messaging.\\ GSM service allows sending data from the app engine or 
		any other backend to android powered devices. GSM is a lightweight push notification based service notifying
		android applications about new data to be fetched from the server or sending messages containing 4KB
		of payload data. GCM manages all aspects of message queuing and delivery of message to target android
		application running on a target device. Applications on the target device need not be running to receive
		messages. This service will wake up the application as long as the application is set up with the proper broadcast 
		receiver and permissions. The application might post a
		notification, display a custom user interface,
		or silently sync data
	\end{abstract}

	\bibliography{../common/seminar}
\bibliographystyle{ieeetr}

\end{document}
